\input{designpatterns.tex}

\chapter{Design}\label{chap:design}

\begin{figure}
\caption{Designklassediagram der viser \klasse{Player} og \klasse{Konto} klasserne.}\label{fig:tikzKlasseKontoPlayer17}
\centering
\tikzumlset{fill class=white, font=\tiny}
\begin{tikzpicture}
\umlclass{Player}{-carColor : int\\-name : String}{+Player(String, int , int)\\+getName() : String\\+getCarColor() : int\\+getKonto() : Konto}
\umlclass[x=5]{Konto}{-balance : int}{+Konto()\\+Konto(int)\\+deposit(int)\\+withdraw(int) : bool\\+getBalance() : int\\+setBalance(int)}
\umlcompo[mult2=1, arg2=-konto,pos2=0.7]{Player}{Konto}
\end{tikzpicture}
\end{figure}

\begin{figure}
\caption{Sekvensdiagram der viser \klasse{Player} og \klasse{Konto} klasserne.}\label{fig:tikzSekvensKontoPlayer17}
\centering
\tikzumlset{fill object=white,font=\tiny}
\begin{tikzpicture}
	\begin{umlseqdiag}
	\umlobject{System}
	\umlcreatecall[x=4, class=Player]{System}{player}
	\umlcreatecall[x=8, class=Konto]{player}{konto}
	
	\begin{umlcall}[op={getName()},return={String name},dt=7]{System}{player}
	\end{umlcall}
	\begin{umlcall}[op={getCarColor()},return={int carColor}]{System}{player}
	\end{umlcall}
	\begin{umlcall}[op={getKonto()},return={Konto konto}]{System}{player}
	\end{umlcall}
	\begin{umlcall}[op={deposit(int)}]{System}{konto}
	\end{umlcall}	
	\begin{umlcall}[op={withdraw(int)},return=bool]{System}{konto}
	\end{umlcall}
	\begin{umlcall}[op={getBalance()},return=int balance]{System}{konto}
	\end{umlcall}
	\begin{umlcall}[op={setBalance(int)}]{System}{konto}
	\end{umlcall}							
	\end{umlseqdiag}
\end{tikzpicture}
\end{figure}
\begin{figure}
\caption{Designklassediagram der viser det meste af spillet. Field og dennes underklasser vises i figur \vref{fig:tikzKlasseFields8}.}\label{fig:tikzKlasseMinusFields8}
\centering
\tikzumlset{fill class=white, font=\tiny}
\begin{tikzpicture}
 \umlclass{Die}{-facevalue : int\\-sides : int\\\umlstatic{-rand : Random}}{+Die()\\+Die(int)\\+Die(Random)\\+rollDie()\\+getFacevalue() : int}
 \umlclass[x=6.5]{MatadorRafleBaeger}{}{+MatadorRafleBaeger()\\+MatadorRafleBaeger(int)\\-createDice()\\+getEns() : bool\\+getSum() : int\\+rollDice()\\+getFacevalues : int[]}
 %\umlclass[x=11]{TestMatadorRafleBaeger}{}{\umlstatic{main(String[])}}
 \umlclass[x=6.5,y=-6]{Game}{-activePlayer : int\\-winpoint : int\\-winner : bool}{+Game()\\-createPlayers(int)\\+startGame()\\-oneRound()\\-endRoundChecks()\\-showStatus(Field)\\-checkWinner()\\-gameEnd()\\-nextPlayer()}
 \umlclass[y=-6]{Player}{-carColor : int\\-name : String}{+Player(String, int, int)\\+getName() : String\\+getCarColor() : int\\+getKonto() : Konto}
 \umlclass[y=-10]{Konto}{-balance : int}{+Konto()\\+Konto(int)\\+deposit(int)\\+withdraw(int) : bool\\+getBalance() : int\\+setBalance(int)}
 \umlclass[x=11, y=-3]{Main}{}{\umlstatic{+main(String[])}}
 \umlclass[y=-11, x=5]{BoundaryToGUI}{}{\umlstatic{+main(String[])}\\\umlstatic{+setDice(MatadorRafleBaeger)}\\\umlstatic{+addPlayer(Player)}\\\umlstatic{+setBalance(Player[])}\\\umlstatic{+setCar(Field, Player)}}
 \umlclass[y=-11,x=11]{BoundaryToPlayer}{\umlstatic{-input : Scanner}}{\umlstatic{+main(String[])}\\\umlstatic{+getPlayerAccept(Player) : bool}\\\umlstatic{-getPlayerInt() : int}\\\umlstatic{+showString(String)}\\\umlstatic{+closeScanner()}\\\umlstatic{+landOnField(Field)}\\\umlstatic{+showStatus(int[],int[])}\\\umlstatic{-getPlayerBalances(Player[]) : int[]}}
 \umlclass[y=-6,x =11.7]{Board}{-boardFields : Field[]}{+Board()\\-initBoard(int)\\+getField(int) : Field}
 
 \umlcompo[mult1=1,arg2=-dice,mult2=0..*]{MatadorRafleBaeger}{Die}
 \umlcompo[mult1=1,mult2=0..1,arg2=-baeger]{Game}{MatadorRafleBaeger}
 \umlcompo[mult1=1,mult2=0..*,arg2=-players]{Game}{Player}
 %\umldep{TestMatadorRafleBaeger}{MatadorRafleBaeger}
 \umldep[]{Main}{Game}
 \umldep{Game}{BoundaryToGUI}
 \umldep{Game}{BoundaryToPlayer}
 \umlcompo[mult2=1, arg2=-konto]{Player}{Konto}
 \umlcompo[mult2=1, arg2=-board]{Game}{Board}
\end{tikzpicture}
\end{figure}

\begin{figure}
\caption{Designklassediagram der viser Board, Field og underklasser af Field.}\label{fig:tikzKlasseFields8}
\centering
\tikzumlset{fill class=white, font=\tiny}
\begin{tikzpicture}

 \umlclass[x=-5,y=-4.5]{Board}{}{+Board()\\-initBoard(int)\\+getField(int) : Field}
 \umlclass[y=-4.5,x=1,type=abstract]{Field}{\#name : String\\\#changeBalance : int\\\#fieldNum : int}{+Field(String, int)\\+Field(String, int, int)\\+getChangeBalance() : int\\+getName() : String\\+getFieldNum() : int}
 \umlclass[y=-11.5,x=-4.5]{Brewery}{}{+Brewery(String, int\\+Brewery(String, int, int))}
 \umlclass[y=-8.5,x=-4.5]{Refuge}{}{+Refuge(String, int\\+Refuge(String, int, int))}
 \umlclass[y=-11.5]{Shipping}{}{+Shipping(String, int\\+Shipping(String, int, int))}
 \umlclass[y=-11.5,x=4.5]{Street}{}{+Street(String, int\\+Street(String, int, int))}
 \umlclass[y=-8.5,x=4.5]{Taxes}{}{+Taxes(String, int\\+Taxes(String, int, int))}
 \umlclass[y=-8.5,type=abstract]{Ownable}{}{+Ownable(String, int\\+Ownable(String, int, int))}
 
 \umlcompo[mult2=0..*,arg2=-boardFields,pos2=1.0,align2=right]{Board}{Field}
 \umlinherit{Ownable}{Field}
 \umlinherit{Taxes}{Field}
 \umlinherit{Refuge}{Field}
 \umlinherit{Brewery}{Ownable}
 \umlinherit{Shipping}{Ownable}
 \umlinherit{Street}{Ownable}
\end{tikzpicture}
\end{figure}

\begin{figure}
\caption{Sekvensdiagram som viser opstart af spillet.}\label{fig:tikzSekvensStartGame8}
\centering
\tikzumlset{fill object=white,font=\tiny}
\begin{tikzpicture}
	\begin{umlseqdiag}
	\umlobject[x=0,class=Main]{Main}
	\umlobject[x=11,y=-8]{BoundaryToPlayer}
		\umlcreatecall[x=2.5,class=Game]{Main}{game}
		\umlcreatecall[x=6,class=MatadorRafleBaeger]{game}{baeger}
		\begin{umlcall}[op=createDice(),padding=1]{baeger}{baeger}
			\umlcreatecall[x=11.5,class=Die,stereo=multi,dt=2]{baeger}{dice}
		\end{umlcall}
		\umlcreatecall[x=7,class=Board,dt=11]{game}{board}
		\begin{umlcall}[op=initBoard(),padding=1]{board}{board}
			\umlcreatecall[class=Field,x=11,stereo=multi,dt=2]{board}{boardFields}
		\end{umlcall}
		\begin{umlcall}[op=createPlayers(),dt=5.5,padding=1]{game}{game}
			\umlcreatecall[class=Player,stereo=multi,x=11,dt=2]{game}{players}
		\end{umlcall}
		\begin{umlcall}[op=startGame(),dt=34.5,padding=-2]{Main}{game}
			\begin{umlcall}[op=showString(String),dt=5]{game}{BoundaryToPlayer}
			\end{umlcall}
			\begin{umlcallself}[op=oneRound(),padding=1,dt=0]{game}
			\end{umlcallself}
		\end{umlcall}
	\end{umlseqdiag}
\end{tikzpicture}
\end{figure}

\begin{figure}
\centering
\caption{Sekvensdiagram som viser en runde af spillet, MRB betyder MatadorRafleBaeger.}\label{fig:tikzSekvensGame8}
\rotatebox{90}{ %Rotates the contents by 90 degrees.
\tikzumlset{fill object=white,font=\tiny}
\begin{tikzpicture}
	\begin{umlseqdiag}
		\umlactor[scale=0.5,y=-0.7]{Spiller 1}
		\umlobject[x=2,class=Game]{game}
		\umlobject[x=4.6]{BoundaryToPlayer}
		\umlobject[x=9.7,class=MRB]{baeger}
		\umlobject[x=11.6,class=Die,stereo=multi]{dice}
		\umlobject[x=7.4]{BoundaryToGUI}
		\umlobject[x=13.5,class=Player,stereo=multi]{players}
		\umlobject[x=15.7,class=Board]{board}
		\umlobject[x=18,class=Field,stereo=multi]{boardFields}
		\umlobject[x=20.5,class=Konto,stereo=multi]{konto}
	
		\begin{umlcallself}[op=oneRound(),dt=3,padding=1]{game}
			\begin{umlcall}[op=showString(),padding=0]{game}{BoundaryToPlayer}
			\end{umlcall}
			\begin{umlcall}[op=getPlayerAccept(),return=bool,padding=1.5]{game}{BoundaryToPlayer}
				\begin{umlcall}[op=Scanner,return=int,padding=1.5,dt=2]{BoundaryToPlayer}{Spiller 1}
				\end{umlcall}
			\end{umlcall}
			\begin{umlcall}[op=rollDice(),dt=2,padding=0]{game}{baeger}
				\begin{umlcall}[op=rollDie(),padding=0,dt=0.5]{baeger}{dice}
				\end{umlcall}
			\end{umlcall}
			\begin{umlcall}[op=getSum(),return=int,padding=1.5]{game}{baeger}
				\begin{umlcall}[op=getFacevalue(), return=int,padding=1.5,dt=0.5]{baeger}{dice}
				\end{umlcall}
			\end{umlcall}
			\begin{umlcall}[op=getField(), return=Field,padding=1.5,dt=2.2]{game}{board}
			\end{umlcall}
			\begin{umlcall}[op=getChangeBalance(), return=int,padding=1.5,dt=2.2]{game}{boardFields}
			\end{umlcall}
			\begin{umlcall}[op=getKonto(), return=Konto,padding=1.5,dt=2.2]{game}{players}
			\end{umlcall}
			\begin{umlcall}[op=deposit(int)/withdraw(int), return=void/bool,padding=1.5,dt=2.2]{game}{konto}
			\end{umlcall}
			\begin{umlcallself}[op=showStatus(),dt=1,padding=1]{game}
				\begin{umlcall}[op={setCar(Field, Player), setDice(MRB), setBalance(Player[])},padding=0]{game}{BoundaryToGUI}
				\end{umlcall}
				\begin{umlcall}[op={showStatus(MRB, Player[]), landOnField(Field)},padding=0]{game}{BoundaryToPlayer}
				\end{umlcall}
			\end{umlcallself}
			\begin{umlcallself}[op=endRoundChecks(),dt=0,padding=1]{game}
				\begin{umlcallself}[op=checkWinner(),dt=1.3,padding=1]{game}
					\begin{umlcall}[op=getBalance(),return=int,padding=1.5,dt=1]{game}{konto}
					\end{umlcall}
					\begin{umlfragment}[type=alt,label=winner,inner xsep=4]
						\begin{umlcall}[op=showString(String),padding=0,dt=1]{game}{BoundaryToPlayer}
						\end{umlcall}
						\begin{umlcallself}[op=gameEnd(),padding=1,dt=1]{game}
							\begin{umlcall}[op=closeScanner(),padding=0,dt=2.2]{game}{BoundaryToPlayer}
							\end{umlcall}
						\end{umlcallself}
					\umlfpart[default]
						\begin{umlcallself}[op=nextPlayer(),padding=1,dt=1]{game}
						\end{umlcallself}
					\end{umlfragment}
				\end{umlcallself}
			\end{umlcallself}
		\end{umlcallself}
	\end{umlseqdiag}
\end{tikzpicture}
}
\end{figure}