\chapter{Indledning[1]}\label{ch:indledning}
Denne rapport er udarbejdet i kurserne 02313 Udviklingsmetoder til IT-systemer, 02312 Indledende programmering og 02314 Indledende programmering på første semester af Diplomineniør IT. Opgaven er del et af tre CDIO opgaver på første semester, det overordnede formål med opgaverne på semesteret er at lave et Matador spil. Denne første opgave fokuserer på at implementere en terningklasse og en MatadorRafleBaeger klasse til at holde terning instanser. Den nærmere kravspecificering til opgaven findes i "CDIO\_opgave\_del11.pdf"\cite{CDIOdel1}, de relevante dele fra denne vil blive omhandlet i rapporten. Opgaven bygger i på undervisning modtaget i ovenstående kurser hvor der blev benyttet bøgerne \cite{umlbook} og \cite{javabook}. Igennem opgaverne vil klasser blive vist på følgende måde \klasse{KlasseNavn} og metoder på følgende måde \metode{metodeNavn()}.

Rapporten er sat med \LaTeX og alle UML diagrammer er lavet med TikZ-UML.

