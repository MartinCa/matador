\chapter{Konklusion[1, 2]}\label{chap:Konklusion}
Formålet med CDIO del 3 var at implementere et klassehierarki over felterne i et matadorspil, med tilhørende abstrakte metoder, til brug for videreudvikling af matadorspillet fra CDIO del 2. De implementerede felter skulle repræsentere de fysiske matadorfelter i funktionalitet. Desuden skulle udvikles en fuld domænemodel over et matadorspil.

De udleverede krav i \cite{CDIOdel3} blev tolket og opstillet i henhold til UP modellen for kravspecificering i \vref{chap:krav}.

Domænemodellen over matadorspillet blev gennemgået og kommenteret i \vref{chap:Design}. I samme kapitel blev de vigtigste ændringer i spillet siden CDIO del 2 gennemgået ved hjælp af designklasse- og sekvensdiagrammer.

I \vref{chap:Test} blev black box testning af spillet gennemgået som en bruger ville opleve systemet.

Koden i Eclipse lever nu op til kravene i \cite{CDIOdel3} og er altså en videreudvikling af CDIO del 2 som implementerer det nye klassehierarki over felterne. I \vref{chap:Implementering} bliver de mest interessante dele af koden gennemgået, med fokus på emner som ikke har været gennemgået i undervisningen.