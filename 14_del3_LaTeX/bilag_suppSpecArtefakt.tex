\chapter{Supplementary Specification artefakt}\label{chap:suppSpec}

\begin{enumerate}
\item \metode{toString} føjes til alle klasser
\item \klasse{Field} skal have attributterne \textit{number} og \textit{name} samt \metode{landOnField()} skal tage \textit{Player} som argument.
\item \klasse{Refuge} skal have attribut \textit{bonus} (kontant beløb).
\item \klasse{Tax} skal have attribut \textit{tax} (kontant beløb).
\item \klasse{Ownable} implementeres som abstrakt klasse der definerer alle felter der kan ejes. 
\item \klasse{Street} simplificeres til kun at have en leje \fixme{hvad har vi gjort/gør vi her?}
\item Hvis der landes på \klasse{Shipping} betales der leje til ejeren af feltet alt efter hvor mange \klasse{Shipping}-felter (\textit{rederier} i formel) denne ejer. Lejen udregnes som: $$leje=2^{(rederier)}*500$$
\item Hvis der landes på \klasse{Brewery} betales der leje til ejeren udfra antallet af øjne i slaget der bragte ham til feltet (\textit{slag} i formel) samt hvor mange bryggerier denne ejer (\textit{bryggerier} i formel. Lejen udregnes som: $$leje=(slag)*(bryggerier)*100$$
\item Modificer CDIO-opgave del 2 så denne anvender det nye arveheiraki.
\item Deklarér klasserne \klasse{Field} og \klasse{Ownable} som abstrakte og deres metoder \metode{landOnField()} og \metode{rent()} skal ligeledes være abstrakte.
\item Lav et program  som indeholder et array af typen Field, som udskriver indholdet af hvert objekt i array’et. Du skal minimum oprette en instans af hver klasse.
\item Domænemodel skal fuldt udvikles og kommenteres.  
\item De beskrevne udvidelser blev implementeret i CDIO-opgave del 2 og er således stadig en del af systemet. 
\end{enumerate}