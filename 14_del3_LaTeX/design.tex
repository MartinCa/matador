\chapter{Design}\label{chap:Design}

\section{Domænemodel[2]}\label{chap:domain}

\begin{figure}
\caption{Domæneklassediagram over matadorspillet.}\label{fig:DomainDia}
\centering
\tikzumlset{fill class=white, font=\tiny}
\begin{tikzpicture}
 \umlclass[x=-4.5,y=-6]{Board}{}{}
 \umlclass[y=-6,type=abstract]{Field}{name\\number}{}
 \umlclass[y=-12,x=2.3]{Brewery}{}{}
 \umlclass[y=-9,x=-4.5]{Refuge}{bonus}{}
 \umlclass[y=-12]{Shipping}{}{}
 \umlclass[y=-12,x=4.6]{Street}{}{}
 \umlclass[y=-9,x=-2.2]{Taxes}{tax}{}
 \umlclass[y=-9,x=4.8,type=abstract]{Ownable}{price\\rent}{}
 \umlclass[y=-2,x=1]{Player}{name}{}
 \umlclass[y=-2,x=4]{Bank}{}{}
 \umlclass[x=-4]{MatadorRafleBaeger}{}{}
 \umlclass[x=-4,y=2]{Die}{value}{}
 \umlclass[y=-5,x=6]{House}{}{}
 \umlclass[x=3]{Money}{}{}
 \umlclass[y=-2,x=-4.5]{Game}{}{}
 \umlclass[y=-3.5,x=-2.5]{Car}{color}{}
 \umlclass[y=0.5]{LuckyCard}{action}{}
 \umlclass[y=-9]{Luck}{}{}
 \umlclass[y=-9,x=2]{Prison}{}{}
 
 \umlcompo[mult2=40]{Board}{Field}
 \umlinherit[mult1=28]{Ownable}{Field}
 \umlinherit[arg1=2]{Taxes}{Field}
 \umlinherit[arg1=3]{Refuge}{Field}
 \umlinherit[mult1=2]{Brewery}{Ownable}
 \umlinherit[mult1=4]{Shipping}{Ownable}
 \umlinherit[mult1=22]{Street}{Ownable}
 \umlassoc{Game}{Board}
 \umlaggreg{Bank}{Field}
 \umlassoc{Game}{MatadorRafleBaeger}
 \umlcompo{MatadorRafleBaeger}{Die}
 \umlaggreg[mult2=*]{Bank}{Money}
 \umlaggreg[mult2=0..*]{Player}{Money}
 \umlaggreg[geometry=-|,mult2=0..5,pos2=1.95]{Street}{House}
 \umlaggreg[name=owns]{Player}{Ownable}
 \umlassoc[stereo=rolls]{Player}{MatadorRafleBaeger}
 \umlassoc{Game}{Player}
 \umlcompo[mult2=1]{Player}{Car}
 \umlassoc[mult1=0..*,mult2=1]{Car}{Field}
 \umlassoc{Game}{LuckyCard}
 \umlassoc[stereo=draws]{Player}{LuckyCard}
 \umlassoc[stereo=buys]{Player}{House}
 \umlinherit[mult1=6]{Luck}{Field}
 \umlinherit[mult1=1]{Prison}{Field}
\end{tikzpicture}
\end{figure}

\subsection{Overblik}\label{chap:domain:sec:overblik}
Domænemodellen i \vref{fig:DomainDia} består af en række kasser repræsenterende matadorspillets forskellige elementer (herefter klasser). Dvs. alle de enkeltgrupperinger gruppen fandt nødvendige til at kunne beskrive et helt matadorspil. I udarbejdelsen af domænemodellen er der trukket meget på den objektorienterede viden opnået gennem kurserne 02312 og 02313. Herunder særligt den anvendte UML syntaks til illustration af multipliciteter, associationer og generelt indbyrdes forhold klasserne imellem. Hver klasses multiplicitet, dvs. antal af forekomster, er angivet med et lille tal uden for klassens navn. Alle klasser har forbindelser til andre klasser. Måden disse forbindelser er tegnet på varierer efter hvordan klassernes indbyrdes forhold er. De steder hvor der står et ord på tværs af en forbindelse angiver dette ord en specificering af denne forbindelses type. Der anvendes en række forskellige syntakser i domænemodellen, hvoraf en af hver vil blive beskrevet i det følgende, samt en mere detaljeret gennemgang af modellen.

\subsection{Modelforklaring}\label{chap:domain:sec:modelForklaring}
Der er 3 centrale elementer i matadorspillet. Disse er den enkelte spiller \klasse{Player}, klassen \klasse{Game} eller spillet som helhed og brættet som udgør den fysiske dimension af spillet. \klasse{Game} skal forstås som en slags abstraktion over regelsættet i matador. Spillet spilles med 2 terninger \klasse{Die} som \klasse{Player} ruller med \klasse{MatadorRafleBaeger}. Det ses at der er 2 terninger i spillet ved det lille 2 tal under klassen \klasse{Die}. Med andre ord dikterer \klasse{Game} at \klasse{Player} ruller \klasse{MatadorRafleBaeger} med 2 terninger i. Derfor er der indbyrdes forbindelser og <<rolls>> på forbindelsen fra \klasse{Player} til \klasse{MatadorRafleBaeger}. \
\linebreak
Hvis der tages udgangspunkt i 1 \klasse{Player} ses det at han har 1 \klasse{Car} som har en farve \textit{color}. Denne \textit{color} står inden i kassen der angiver \klasse{Car} og er således en attribut. Altså en egenskab som \klasse{Car} har. Det ses også at \klasse{Player} har \klasse{Money} og multipliciteten "0..*" betyder at spilleren kan have fra 0 til uendeligt mange penge. Der er selvfølgeligt nogle fysiske begrænsninger på dette og * kunne også skiftes ud med den sum man skal have for at vinde spillet. \klasse{Player} har også forbindelse til \klasse{LuckyCard} hvorpå der står <<draw>>. \klasse{Player} trækker altså et \klasse{LuckyCard}. Dette sker når der landes på feltet luck. Der er taget beslutning om ikke at tilføje en forbindelse fra \klasse{LuckyCard} til \klasse{Luck} af hensyn til overskueligheden, men denne kunne meget vel også tilføjes. \klasse{Player} har også en forbindlese til \klasse{Game}. Denne skyldes det faktum at \klasse{Game} er den regeldefinerende klasser og således bestemmer restriktioner for alt i spillet. Således kunne man med rette argumentere for at \klasse{Game} skulle have forbindelse til alle dele af spillet, Det er dog valgt kun at medtage de vigtigste forbindelser. Ydermere har \klasse{Player} forbindelse til \klasse{House} med <<buys>> på. Dette indikerer at \klasse{Player} kan købe \klasse{House}. Den videre forbindelse fra \klasse{House} til \klasse{Street} viser at disse hører til på en \klasse{Street}. \
Under \klasse{Game} er en forbindelse til \klasse{Board}. Dette er selve spillepladen. Der er kun 1 spilleplade per spil og denne består endvidere af 40 \klasse{Fields}. Hvert \klasse{Field} har et \textit{name} og \textit{number}. Der kan stå flere biler på samme \klasse{Field}, men hver bil kan kun stå på et \klasse{Field} ad gangen. Dette ses ved multipliciteterne på forbindelsen mellem de 2 klasser. \
De 40 \klasse{Field}'s som spillepladen består af er som nævnt inddelt i underklasser. Hver underklasse har en speciel funktion. Multipliciteterne angiver hvor mange der er af hvert felt. Underklassen \klasse{Ownable} er en abstraktion over de \klasse{Field}'s der har den fælles egenskab at de kan ejes. Udover at have \textit{name} og \textit{number} attributterne som tidligere angivet har de også attributterne \textit{price} og \textit{rent}. I standard versionen af et matadorspil findes der også et skøde til hver \klasse{Ownable}. Der er taget beslutning om at udelade dette i modellen da skøder ikke er nødvendige for at illustrere spillet, men de kunne naturligvis indføres.           

\section{Designklassediagram[2]}\label{sec:desKlas}
\subsection{Generelt}
\vref{fig:DesignKlasseChanged} viser et udsnit af det totale spil. Det er således kun de, for denne 3. del af CDIO projektet særligt interessante, klasser der er afbildet. For designklassediagram og beskrivelse for de resterende dele henvises til \cite{CDIORapportDel2}. I det følgende vil overordnede designbeslutninger beroende på gruppens viden om designpatterns og principper blive behandlet.

\begin{figure}
\caption{Designklassediagram over ændrede klasser i matadorspillet.}\label{fig:DesignKlasseChanged}
\centering
\tikzumlset{fill class=white, font=\tiny}
\begin{tikzpicture}

 \umlclass[x=-5,y=0.5]{Game}{-activePlayer : int\\-winpoint : int\\-baeger : MatadorRafleBaeger\\\umlstatic{-game : Game}}{\umlstatic{+getGame()}\\-createPlayers(int)\\+startGame()\\-oneRound()\\-optToBuy(Field, Player)\\-endRoundChecks()\\-showStatus(Field)\\-checkWinner()\\-gameEnd()\\-nextPlayer()\\+getBaeger() : MatadorRafleBaeger}
 \umlclass[x=2,y=0.5]{Player}{-carColor : int\\-name : String\\-konto : Konto\\-ownedFields : List<Ownable>\\-loser : bool}{+Player(String, int, int)\\+getName() : String\\+getCarColor() : int\\+getKonto() : Konto\\+buyField(Ownable)\\+getOwnedFields() : List<Ownable>\\+isLoser() : bool\\+setLoser()}
 \umlclass[x=-5,y=-4.5]{Board}{}{-initBoard(int)\\+getField(int) : Field}
 \umlclass[y=-4.5,x=1,type=abstract]{Field}{\#name : String\\\#fieldNum : int}{\textit{+landOnField(Player)}\\+getName() : String\\+getFieldNum() : int}
 \umlclass[y=-11.5,x=-4.5]{Brewery}{}{\#rent() : int\\+landOnField(Player)}
 \umlclass[y=-8.5,x=-4.5]{Refuge}{-bonus : int}{+landOnField(Player)}
 \umlclass[y=-11.5]{Shipping}{-basisFare : int}{\#rent() : int\\+landOnField(Player)}
 \umlclass[y=-11.5,x=4.5]{Street}{-rent : int}{\#rent() : int\\+landOnField(Player)}
 \umlclass[y=-8.5,x=4.5]{Taxes}{}{+landOnField(Player)}
 \umlclass[y=-8.2,type=abstract]{Ownable}{\#price : int\\\#owner : Player}{+getPrice() : int\\+getOwner() : Player\\+setOwner(Player)\\\textit{\#rent() : int}}
 
 \umlcompo[mult2=0..*,arg2=-boardFields,pos2=1.0,align2=right]{Board}{Field}
 \umlinherit{Ownable}{Field}
 \umlinherit{Taxes}{Field}
 \umlinherit{Refuge}{Field}
 \umlinherit{Brewery}{Ownable}
 \umlinherit{Shipping}{Ownable}
 \umlinherit{Street}{Ownable}
 \umlaggreg[arg2=-players,mult2=2,align2=right,pos2=1]{Game}{Player}
 \umlcompo[arg2=-board,mult2=1]{Game}{Board}
\end{tikzpicture}
\end{figure}

\subsection{Arv og polymorfi}
I denne del af pojektet er det særligt interssant at kigge på forholdet mellem \klasse{Field} og de klasser der arver fra denne.\klasse{Field} er en abstrakt klasse og det ses at klasserne \klasse{Refuge}, \klasse{Taxes} og \klasse{Ownable} arver fra \klasse{Field}. \klasse{Ownable} er også en abstrakt klasse som så igen har underklasserne \klasse{Brewery}, \klasse{Shipping} og \klasse{Street}. Alle underklasserne i dette arvehieraki har hver deres polymofiske implementation af de, i de abstrakte klasser definerede, abstrakte metoder. 

\subsection{Singleton}\label{sec:desSingleton}
I spillet er der kun behov for en instans af klassen \klasse{Game} og \klasse{Brewery} har behov for at kunne hente værdien af terningerne, til udregning af leje, er der taget beslutning om at anvende \dpattern{Singleton} ved oprettelse af \klasse{Game}. Dette ses også i \vref{fig:DesignKlasseChanged} ved at angivelserne af atributten \textit{game} og \metode{getGame} begge er markeret som statiske. På denne måde kan \klasse{Brewery} hente summen af terningerne ud gennem den statiske adgang til \klasse{Game}. 

\section{Sekvensdiagrammer[1]}
For at vise hvordan spillet fungerer er der produceret to sekvensdiagrammer. Disse to sekvensdiagrammer viser de dele af systemet som har undergået betydningsfulde ændringer siden del 2. Der er tale om implementeringen af metoderne \metode{optToBuy(Field, Player)} og \metode{landOnField(Player)} i henholdsvis \klasse{Game} og \klasse{Field}s underklasser.

\metode{optToBuy(Field, Player)} kaldes hver gang en spiller lander på et felt, sekvensdiagrammet ses i \vref{fig:SekvensOptToBuy}. Metoden undersøger ved hjælp af operatøren instanceof, beskrevet i \vref{sec:instanceof}, om feltet kan købes. Derefter undersøges ved et kald til \metode{getOwner()} om der allerede er en ejer.

Hvis feltet kan købes undersøges om spilleren har nok penge i kontoen til at købe feltet, hvis spilleren har nok penge gives muligheden for at købe  feltet.

Hvis spilleren accepterer at købe feltet trækkes pengene fra spilleren konto og feltet tilføjes til spillerens liste over ejede felter. Derudover sættes spilleren også som ejer af feltet i feltet.

\begin{figure}
\caption{Sekvensdiagram over \metode{optToBuy(Field, Player)}.}
\label{fig:SekvensOptToBuy}
\centering
\tikzumlset{fill object=white,font=\tiny}
\begin{tikzpicture}
	\begin{umlseqdiag}
		\umlactor[x=-4]{Player}
		\umlobject[x=-2.5,class=Game]{game}
		\umlobject[x=0,class=Ownable]{ownable}
		\umlobject[x=2.7,class=Player,stereo=multi]{player}
		\umlobject[x=5,class=Konto]{konto}
		\umlobject[x=8]{BoundaryToPlayer}
		
		\begin{umlcall}[op={optToBuy(actField, actPlayer)}]{game}{game}
			\begin{umlcall}[op=isInstance(actField),return=bool ownable]{game}{ownable}
			
			\end{umlcall}
			\begin{umlfragment}[type=opt,label=ownable,inner xsep=0.2]
				\begin{umlcall}[op=getOwner(),return=Player owner]{game}{ownable}
				\end{umlcall}
				\begin{umlfragment}[type=opt,label={owner!=null},inner xsep=0.3]
					\begin{umlcall}[op=getKonto(),return=konto]{game}{player}				
					\end{umlcall}
					\begin{umlcall}[op=getBalance(),return=int balance]{game}{konto}					
					\end{umlcall}
					\begin{umlcall}[op=getPrice(),return=int price]{game}{ownable}
					\end{umlcall}
					\begin{umlfragment}[type=opt,label={balance>=price},inner xsep=0.3]
						\begin{umlcall}[op=opToBuy(),return= bool optToBuy]{game}{BoundaryToPlayer}
							\begin{umlcall}[op=getPlayerInt,return=int]{BoundaryToPlayer}{Player}
							
							\end{umlcall}
						
						\end{umlcall}
						\begin{umlfragment}[type=opt,label=optToBuy,inner xsep=0.2]
							\begin{umlcall}[op=buyField(actField),return=.]{game}{player}
								\begin{umlcall}[op=getPrice(),return=int price]{player}{ownable}
								
								\end{umlcall}
								\begin{umlcall}[op=withdraw(price)]{player}{konto}
								
								\end{umlcall}
								\begin{umlcallself}[op={ownedFields.add(actField)}]{player}
								
								\end{umlcallself}
							
							\end{umlcall}
						\end{umlfragment}
					\end{umlfragment}
				\end{umlfragment}
			\end{umlfragment}
		\end{umlcall}
	\end{umlseqdiag}
\end{tikzpicture}
\end{figure}

\metode{landOnField(Player)} håndterer den begivenhed at en spillet lander på et felt. Implementeringen af \metode{landOnField(Player)} varierer for de forskellige typer af felter. Fælles er dog at de alle sammen sørger for at lave de nødvendige transaktioner med spillerens konto som resultat af at lande på feltet. Som eksempel er vist metoden for felter af typen \klasse{Ownable} da metoden er mere kompliceret for disse felter da der eventuelt skal overføres penge mellem spillere.

Sekvensdiagrammet ses i \vref{fig:SekvensLandOnField}, metoden starter med et kald fra \klasse{Game}. Først og fremmest ses at hvis spilleren der lander på feltet også ejer feltet sker der intet. Ellers beregnes lejen(rent), prisen for at lande på feltet. Lejen trækkkes fra spillerens konto og hvis spilleren har nok penge til at betale lejen vil lejen blive overført til ejeren af feltet, hvis der er en sådan. Hvis spilleren ikke har nok penge til at betale lejen sættes spilleren som taber i spillet.

Efter dette gives kontrollen tilbage til \klasse{Game}.


\begin{figure}
\caption{Sekvensdiagram over \metode{landOnField(Player)}.}
\label{fig:SekvensLandOnField}
\centering
\tikzumlset{fill object=white,font=\tiny}
\begin{tikzpicture}
	\begin{umlseqdiag}
		\umlobject[x=0,class=Field,stereo=multi]{field}
		\umlobject[x=4,class=Player,stereo=multi]{player}
		\umlobject[x=8,class=Konto]{konto}
		
		\begin{umlcallself}[op={rent=rent()}]{field}
		\end{umlcallself}
		\begin{umlcall}[op=getKonto(),return=konto]{field}{player}
		\end{umlcall}
		\begin{umlcall}[op=withdraw(rent),return=bool success]{field}{konto}
		\end{umlcall}
		\begin{umlfragment}[type=alt,label=owner]
			\begin{umlcall}[op=deposit(rent),dt=4]{field}{player}
			\end{umlcall}
		\end{umlfragment}
		\begin{umlfragment}[type=alt,label=not success]
			\begin{umlcall}[op=setLoser(),dt=6]{field}{player}
			\end{umlcall}
		\end{umlfragment}
	\end{umlseqdiag}
\end{tikzpicture}
\end{figure}