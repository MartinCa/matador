\chapter{Konklusion[1, 2]}\label{chap:konklusion}
Formålet med CDIO del 2 var at implementere klasserne \klasse{Player} og \klasse{Konto} og to krævede metoder. Med disse klasser samt den tidligere \klasse{MatadorRafleBaeger}, ønskedes udviklet et lille spil. Systemet skal desuden dokumenteres med UML designklassediagrammer og designsekvensdiagrammer.

For at udvide systemet med den nye funktionalitet ses i \vref{chap:krav}, på først hvilket udgangspunkt der er fra det tidligere projekt, dernæst hvilke krav der er til dette projekt. De fleste af klasserne fra det tidligere projekt kunne genbruges i dette projekt, hvilket har mindsket arbejdet med implementeringen. Dette har givet mulig for at forberede systemet til senere iterationer.

Designet af systemet blev gennemgået ved hjælp af designklassediagrammer og designsekvensdiagrammer.

Koden i Eclipse indeholder nu et spil der lever op til kravene stillet i opgavebeskrivelsen \cite{CDIOdel2} med alle udvidelsesmulighederne tilføjet. Derudover er udviklet en unittest for en enkelt klasse for at få erfaring med udviklingen af unittests.