\chapter{Krav[1]}\label{chap:krav}

\section{Udgangspunkt}\label{sec:krav:kravUdgangspunkt}
Som udgangspunkt for starten af dette projekt bruges koden som Martin havde skrevet i forbindelse med CDIO del 1 i gruppe 14. Derfor var det kodemæssige udgangspunkt at kodebasen levede op til kravene fra CDIO del 1 som fortolket af gruppe 14.

En del af designet af løsningen til CDIO del 2 er altså at finde ud af hvilke dele af koden til del 1 der kan bruges i del 2. Martin valgte under programmeringen af del 1 af opdele spillet i forskellige klasser som havde forskelligt ansvar i forhold til at implementere spillet i del 1. På grund af den måde spillet var kodet på kunne de fleste af klasserne genbruges med få modifikationer. Selve spillogikken skal sandsynligvis undergå ændringer eftersom spillet i del 2 ikke er det samme som i del 1.

I del 1 havde Martin kodet følgende klasser som havde med selve spillet at gøre:
\begin{itemize}
\item \klasse{Die}
\item \klasse{MatadorRafleBaeger}
\item \klasse{Player}
\item \klasse{Game}
\item \klasse{GameController}
\item \klasse{BoundaryToPlayer}
\item \klasse{BoundaryToGUI}
\item \klasse{Main}
\end{itemize}

Af disse klasser skal især \klasse{Game} undergå store ændringer da det blev valgt at starte på ny med spillogikken i del 2. Det er dog mere interessant at overveje om de andre klasser i systemet kan bruges i dette spil. Af speciel interesse er især klasserne \klasse{Die}, \klasse{MatadorRafleBaeger} og \klasse{Player}. Hvis disse klasser ikke afhænger af \klasse{Game} for deres funktion vil de sandsynligvis kunne anvendes mere eller mindre uændret i dette spil. Heldigvis blev disse klasser kodet med hensyn til lav kobling (Low Coupling, se \vref{Chapter:Design patterns:Anvendelse:Low Coupling}) og ingen af klasserne kender overhovedet til \klasse{Game}. Klasserne fra del 1 kan derfor genbruges i dette spil hvor \klasse{Game} sandsynligvis vil være ændret. \klasse{Player} indeholder informationer der ikke er nødvendige i dette spil, ændringerne til \klasse{Player} vil blive behandlet senere i \vref{chap:design}.

Med hensyn til resten af klasserne vil Boundary klasserne i stor stil kunne genbruges. Dette spil kommunikerer nemlig også med den samme udleverede GUI og konsollen.

Udgangspunktet for projektet er nu mere klart og kravene til det nye spil undersøges i det følgende afsnit.

\section{Kravspecifikation for del 2}\label{sec:krav:kravSpec}
Kravene til CDIO del 2 bliver stillet op ved hjælp af UP modellen for en kravspecifikation som præsenteret i kursus 02313. Den udarbejdede kravspecifikation bliver præsenteret i dette afsnit, hvor der er uklarheder i den udleverede tekst er de tolket efter bedste evne. I en virkelig opgave skulle uklarheder selvfølgelig afklares i samarbejde med kunden i Incpetion fasen. Kilde til dette afsnit er \cite{CDIOdel2}, som er opgavebeskrivelsen for CDIO del 2.

\subsection{Formål med systemet}\label{sec:krav:kravSpec:formaal}
Formålet med systemet er at der skal implementeres to klasser \klasse{Player} og \klasse{Konto}, som henholdsvis beskriver en matadorspiller og en spillers kontantbeholdning. Kravene til begge klasser er at de skal indeholde passende \metode{get}, \metode{set} og \metode{toString} metoder. Derudover skal \klasse{Konto} indeholde to metoder: \metode{deposit(amount)} og \metode{withdraw(amount)}.

\metode{deposit(amount)} skal tilføje amount til spillerens kontantbeholdning.

\metode{withdraw(amount)} skal fjerne amount fra spillerens kontantbeholdning dog med den lille tilføjelse at kontantbeholdningen ikke må kunne gå i minus og metode skal fortælle om dette.

Ved at bruge disse klasser samt \klasse{MatadorRafleBaeger} skal designes et lille spil. I spillet ønskes der benyttet en udleveret GUI som viser status af spillet.

\subsection{Funktionelle krav}\label{sec:krav:kravSpec:funkKrav}

\subsubsection{Use case diagram}
I forbindelse med dette spil er der kun en menneskelig og en system aktør, henholdsvis spilleren og den udleverede GUI. Der er identificeret to use cases, startGame og rollDie som bruges til henholdsvis at starte spillet og for at rulle terningerne i spillet. Begge use cases bruges af spilleren og kommunikere med GUI. Use case diagrammet kan ses i \vref{fig:tikzUseCase}.

\begin{figure}
\caption{Use case diagram.}\label{fig:tikzUseCase}
\centering
\tikzumlset{fill usecase=white,font=\tiny}
\begin{tikzpicture}
	\umlactor[x=0, y=-0.5]{Spiller}
	\umlactor[x=6,y=-0.5]{<<system>>GUI}
	\begin{umlsystem}[]{Spillet}
		\umlusecase[x=3, y=0]{rollDie}
		\umlusecase[x=3, y=1]{startGame}
		\umlusecase[x=3, y=-1]{optToBuy}
		\umlusecase[x=3, y=-2]{landOnField}
	\end{umlsystem}
	
	\umlassoc{Spiller}{usecase-1}
	\umlassoc{Spiller}{usecase-2}
	\umlassoc{Spiller}{usecase-3}
	\umlassoc{Spiller}{usecase-4}
	\umlassoc{usecase-1}{<<system>>GUI}
	\umlassoc{usecase-2}{<<system>>GUI}
	\umlassoc{usecase-3}{<<system>>GUI}
	\umlassoc{usecase-4}{<<system>>GUI}
\end{tikzpicture}
\end{figure}

\subsubsection{Use case beskrivelser}

\subsubsection{startGame}

\paragraph{Deltagende aktører:} Initieres af spilleren og interagerer med GUI.

\begin{enumerate}
\item Spilleren starter spillet.
\item Systemet sætter systemet op til at køre spillet.
\item Systemet giver kontrollen tilbage til spilleren og han kan nu rulle terningerne, se use case rollDie.
\end{enumerate}

\subsubsection{rollDie}
\paragraph{Deltagende aktører:} Initieres af spilleren, interagerer med GUI.

\paragraph{Pre konditioner:} Spillet er startet jævnfør use case startGame.

Standard forløb uden vinder:
\begin{enumerate}
\item Spilleren har kontrollen og systemet er sat op og klar til at spille, se use case startGame.
\item Spilleren ruller terningerne.
\item Systemet ruller terningerne som repræsenteret i systemet.
\item Systemet kontrollerer hvilket felt spilleren er landet på.
\item Systemet undersøger hvordan spillerens kontantbeholdning påvirkes.
\item Systemet opdaterer spillerens kontantbeholdning.
\item Systemet præsenterer status af spillet for spilleren.
\item Systemet kontrollerer om der er en vinder.
\end{enumerate}
Trin 2 - 8 gentages indtil der bliver fundet en vinder, hver gang trin 2 nås skiftes til den anden spiller.

\begin{enumerate}[8a.]
\item Der findes en vinder.
	\begin{enumerate}[1.]
	\item Systemet præsenterer vinderen af spillet.
	\item Systemet lukker spillet ned.
	\end{enumerate}
\end{enumerate}

\subsubsection{Aktørbeskrivelser}

\paragraph{Spilleren} Det forventes at spilleren starter spillet og at spilleren ruller med terningerne når systemet anmoder om dette. Derudover præsenterer systemet oplysninger for spilleren for at oplyse om status af spillet.

\paragraph{GUI} Det forventes at GUI er i stand til at præsentere de nødvendige oplysninger for at vise spillets status til spilleren.
