\chapter{Indledning[1]}\label{chap:indledning}
Denne rapport er udarbejdet i kurserne 02313 Udviklingsmetoder til IT-systemer, 02312 Indledende programmering på første semester af Diplomigneniør IT. Opgaven er del tre af tre CDIO opgaver på første semester, det overordnede formål med opgaverne på semesteret er at lave et Matador spil. Denne tredje opgave fokuserer på at implementere et klassehieraki over felterne i et matadorspil og modificere spillet fra del 2. Den nærmere kravspecificering til opgaven findes i "CDIO\_opgave\_del3.pdf"\cite{CDIOdel3}, kravene bliver behandlet i \vref{chap:krav}. Opgaven bygger på undervisning modtaget i ovenstående kurser hvor der blev benyttet bøgerne \cite{umlbook} og \cite{javabook}. Igennem opgaverne vil klasser blive vist på følgende måde \klasse{KlasseNavn}, metoder \metode{metodeNavn()} og design patterns \dpattern{DesignPattern}.

Rapporten er sat med \LaTeX, UML diagrammer er lavet med Ti\emph{k}Z-UML og andre figurer er lavet med Ti\emph{k}Z og \textsc{PGF}.

Bemærk venligst at Martin Caspersen tidligere var med i gruppe 14 og Jesper Engholm Baltzersen tidligere var med i gruppe 17. Begge skrev størstedelen af rapporterne til CDIO del 1 og derfor kan der forekomme passager i denne rapport som minder om passager fra gruppe 14 og 17s rapporter til CDIO del 1. Kodebasen der er benyttet som udgangspunkt for del 2 er den kode som Gruppe 14 afleverede til del 1.

\section{Ansvarfordeling}\label{sec:indledning:ansvarsfordeling}

Ansvarsfordelingen i opgaven er anført i overskrifterne til de forskellige afsnit og benytter følgende numre for at beskrive gruppens medlemmer:
\begin{enumerate}
\item Martin Caspersen
\item Jesper Engholm Baltzersen
\end{enumerate}
Ansvarsfordelingen overskues nemmest i indholdsfortegnelsen. Hvis der er nummer på et af hovedafsnittene og ikke nogen numre på underafsnit er det pågældende medlem ansvarlig for alle underafsnittene også.

Alle figurer og diagrammerne er udarbejdet af Martin, men både Martin og Jesper er ansvarlige for indholdet af disse.

Ansvaret for kodningen af de forskellige klasser er beskrevet herunder:
\begin{itemize}
\item Die [1, 2]
\item MatadorRafleBaeger [1, 2]
\item Game [1, 2]
\item Player [1, 2]
\item BoundaryToGUI [1, 2]
\item BoundaryToPlayer [1, 2]
\item Main [1, 2]
\item Board [1, 2]
\item Field og underklasser [1, 2]
\item TestDie [1, 2]
\item GameBoard [1, 2]
\end{itemize}